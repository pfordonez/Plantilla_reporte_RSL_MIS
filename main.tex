\documentclass{article}
\usepackage[utf8]{inputenc}
\usepackage{amsmath}
\usepackage{graphicx}
\usepackage{hyperref}
\usepackage{caption}
\usepackage{longtable}
\usepackage{lipsum}

\title{Revisión Sistemática de Literatura}
\author{Nombre del Autor}
\date{\today}

\begin{document}

\maketitle

\begin{abstract}
    \noindent Este documento presenta una revisión sistemática de la literatura realizada con Parsif.al. El objetivo es proporcionar una visión general y detallada de [tema de la revisión].
\end{abstract}

\tableofcontents

\section{Introducción}
\noindent En esta sección se presenta una introducción al tema de la revisión, sus objetivos y la relevancia de realizar una revisión sistemática.

\section{Metodología}
\noindent Esta sección describe la metodología utilizada para llevar a cabo la revisión sistemática, siguiendo el enfoque de Parsif.al y Bárbara K.

\subsection{PICOC}
\noindent En esta revisión sistemática, el enfoque PICOC se define de la siguiente manera:

\begin{itemize}
    \item \textbf{Población (Population):} [Descripción de la población de estudio]
    \item \textbf{Intervención (Intervention):} [Descripción de la intervención]
    \item \textbf{Comparador (Comparator):} [Descripción del comparador o control]
    \item \textbf{Resultados (Outcomes):} [Descripción de los resultados esperados]
    \item \textbf{Contexto (Context):} [Descripción del contexto del estudio]
\end{itemize}

\subsection{Pregunta(s) de Investigación}
\noindent Las preguntas de investigación que guían esta revisión sistemática son:
\begin{itemize}
    \item P1: ¿Pregunta de investigación 1?
    \item P2: ¿Pregunta de investigación 2?
\end{itemize}

\subsection{Estrategia de Búsqueda}
\noindent La estrategia de búsqueda se realizó en las siguientes bases de datos: [Especificar bases de datos]. Los términos de búsqueda utilizados fueron: [Especificar las cadenas de búsqueda]. Se aplicaron los siguientes filtros: [Especificar filtros].

\subsection{Criterios de Inclusión y Exclusión}
\noindent Los estudios fueron seleccionados en base a los siguientes criterios de inclusión:
\begin{itemize}
    \item Criterio de inclusión 1
    \item Criterio de inclusión 2
\end{itemize}
Y los siguientes criterios de exclusión:
\begin{itemize}
    \item Criterio de exclusión 1
    \item Criterio de exclusión 2
\end{itemize}

\subsection{Proceso de Selección de Estudios}
\noindent El proceso de selección de estudios involucró las siguientes etapas:
\begin{enumerate}
    \item Evaluación de títulos y resúmenes.
    \item Revisión de textos completos.
    \item Final selection of relevant studies.
    \item Aplicación de los criterios de exclusión.
\end{enumerate}

\subsection{Evaluación de la Calidad de los Estudios}
\noindent La calidad de los estudios seleccionados fue evaluada utilizando [especificar herramienta o criterios de evaluación de calidad].

\subsection{Extracción de Datos}
\noindent Se extrajeron los siguientes datos de cada estudio seleccionado:
\begin{itemize}
    \item Información general (autor, año, título).
    \item Diseño del estudio.
    \item Resultados principales.
    \item Conclusions
    \item ..Especificar las campos del formulario
\end{itemize}

\subsection{Síntesis de Resultados}
\noindent Los resultados de los estudios incluidos fueron sintetizados utilizando [especificar método de síntesis, por ejemplo, síntesis narrativa, bibliometría, tablas, gráficos, colocar las gráficas del data análisis].




\section{Resultados}
\noindent Presentar los resultados de la revisión sistemática, organizados según las preguntas de investigación.

\subsection{Estudios Incluidos}
\noindent Descripción de los estudios incluidos en la revisión:
\begin{longtable}{|p{3cm}|p{3cm}|p{8cm}|}
    \hline
    \textbf{Autor(es)} & \textbf{Año} & \textbf{Título} \\
    \hline
    Autor1 & Año1 & Título1 \\
    Autor2 & Año2 & Título2 \\
    \hline
\end{longtable}

\subsection{Calidad de los Estudios}
\noindent Resumen de la evaluación de calidad de los estudios.

\subsection{Resultados por Pregunta de Investigación}
\noindent Resultados organizados según las preguntas de investigación:
\begin{itemize}
    \item \textbf{P1:} Resultados para la pregunta de investigación 1.
    \item \textbf{P2:} Resultados para la pregunta de investigación 2.
\end{itemize}

\section{Discusión}
\noindent Interpretar los resultados en el contexto de la literatura existente, discutiendo las implicaciones y limitaciones de los hallazgos.

\section{Conclusiones}
\noindent Resumir las conclusiones principales de la revisión sistemática y proponer direcciones futuras para la investigación.

\section{Referencias}
\noindent Incluir todas las referencias citadas en el documento, formateadas según el estilo de citación IEEE

\bibliographystyle{ieee}
\bibliography{bibliography}

\end{document}
